\documentclass[11pt,a4paper,BCOR10mm,bibtotocnumbered]{scrbook}

%% \usepackage{abstract}          % typesetting of a abstract environment

\usepackage{acronym}          % acronyms
\usepackage{amscd}            % Kommutative Diagramme
\usepackage{amsmath}          % mathematics
\usepackage{amsfonts}         % mathematics
\usepackage{array}            % tabels
\usepackage{fancyvrb}         % Umgebung f�r Programm-Listings
\usepackage{float}            % mehr Positionen von Float-Umgebungen
\usepackage[T1]{fontenc}      % Fonteinstellung
\usepackage{graphicx}         % Laden von Bildern
%\usepackage{glossary}         % Erstellen eines Glossars DONT EVER USE THIS
\usepackage[nonumberlist]{glossaries}
\usepackage{hyperref}         % Querreferenzen und Links
\usepackage[latin1]{inputenc} % Zeichenkodierung
\usepackage{keystroke}        % Tastendarstellung
\usepackage{lscape}           % Landscape 
\usepackage{makeidx}          % Indexerstellung
\usepackage{mflogo}           % LaTeX-Logos
\usepackage{multicol}         % mehrspaltiger Satz
\usepackage{multirow}
\usepackage[numbers]{natbib}  % Autor-Datum Zitierungen
%\usepackage{ngerman}          % Zur Verwendung der deutschen Sprache
\usepackage{rotating}         % Rotieren von Bildern und Texten
\usepackage{syntax}           % Syntax-Diagramme
\usepackage{typearea}         % Satzspiegelberechnung
\usepackage{url}              % Hyperlinks
\usepackage{xspace}           % Korrekter Leerraum nach Befehlen
\usepackage{listings}         % Auflistung von Ausschnitten von Programm-Code
\usepackage[T1]{fontenc}      % ben�tigt von luximono
\usepackage{subfigure}        % Unterabbildungen
\usepackage{textcomp}         % Verschiedene Zeichen im textmode
\usepackage{xcolor}
\usepackage{courier}
 


\usepackage[bachelorproject,english]{rtsstud}          % Lehrstuhl-Spezifikationen
\usepackage{rtsabstr}         % Zusammenfassung

\usepackage{mystyle}          % eigene LaTeX-Spezifikationen
  

\title{Configurations and Automated Execution in the KIELER Execution Manager}
\subtitle{~}
\author{cand.~inform.~S\"oren Hansen}
\date{\today}
\advisor{%
  \renewcommand{\thefootnote}{\fnsymbol{footnote}}%
  Christian Motika\\
}

% {%
% \renewcommand{\thefootnote}{\fnsymbol{footnote}}
% \protect\footnotemark
% \footnotetext[1]{Das ist eine Anmerkung zur Person.}
% }

\makeindex


\begin{document}

\frontmatter
\maketitle
% \assertion
% \begin{abstract}
Execution Manager is used in KIELER as a framework to plug-in DataComponents for various tasks. Examples are:
\begin{itemize}
 \item Simulation Engines
 \item Model Visualizers
 \item Environment Visualizers
 \item Validators
 \item User Input Facilities
 \item Trace Recording Facilities 
\end{itemize}

These DataComponents can be executed using a graphical user interface (GUI). 
The scheduling order can also be defined by this GUI as well as other settings like a step/tick duration and properties of DataComponents.

In order to make it more comfortable for the user, preference pages for the Execution Manager could offer the following:
\begin{itemize}
 \item Configure a default configuration
 \item Including timeouts, debug level, scheduling, other settings
 \item Make these things specific for a diagram type
 \item Provide a mechanism to access some last used configurations/schedules 
\end{itemize}

For information about KIEM see - http://rtsys.informatik.uni-kiel.de/trac/kieler/wiki/Projects/KIEM

\end{abstract}


%% Inhaltsverzeichnis
\tableofcontents

%% Inhaltsverzeichnis
%\listoftables

%% Inhaltsverzeichnis
\listoffigures

%% Verzeichnis der Auflistungen
% \lstlistoflistings


%% Verzeichnis der Abk�rzungen

%% Abbreviations
\chapter*{Abbreviations}
\begin{acronym}
 \acro{API}{Application Programming Interface}
 \acro{GUI}{Graphical User Interface}
 \acro{IDE}{Integrated Development Environment}

 \acro{KIEL}{Kiel Integrated Environment for Layout}
 \acro{KIELER}{Kiel Integrated Environment for Layout Eclipse Rich Client}
 \acro{KIEM}{\ac{KIELER} Execution Manager}
 \acro{KIEMAuto}{Automated Executions for the \ac{KIEM}}
 \acro{KIEMConfig}{Configurations for the \ac{KIEM}}

 \acro{UI}User Interface
\end{acronym}


%%% Local Variables: 
%%% mode: latex
%%% TeX-master: "paper"
%%% End: 


%% Verzeichnis der Akronyme
% 
\chapter*{Verzeichnis der Akronyme}
\begin{acronym}[KIEL]
 \acro{KIEL}{Kiel Integrated Environment for Layout}
\end{acronym}

%%% Local Variables: 
%%% mode: latex
%%% TeX-master: "paper"
%%% End: 


%%% Local Variables: 
%%% mode: latex
%%% TeX-master: "paper"
%%% End: 
% 
\chapter*{Verzeichnis der Akronyme}
\begin{acronym}[KIEL]
 \acro{KIEL}{Kiel Integrated Environment for Layout}
\end{acronym}

%%% Local Variables: 
%%% mode: latex
%%% TeX-master: "paper"
%%% End: 


\mainmatter

% \part{Erster Teil} % Teile nur bei sehr langen Arbeiten

\chapter{Introduction}
\label{chapter:introduction}
\section{Description of KIELER}
\label{sec:intro/Kieler}
\begin{itemize}
 \item layouter, structure based editing
 \item synchcharts
 \item simulater
 \item execution manager
\end{itemize}
\section{Description of the KIEM}
\label{sec:intro/Kiem}
Execution Manager is used in KIELER as a framework to plug-in DataComponents for various tasks. Examples are:
\begin{itemize}
 \item Simulation Engines
 \item Model Visualizers
 \item Environment Visualizers
 \item Validators
 \item User Input Facilities
 \item Trace Recording Facilities 
\end{itemize}

These DataComponents can be executed using a graphical user interface (GUI). 
The scheduling order can also be defined by this GUI as well as other settings like a step/tick duration and properties of DataComponents.
For information about \ac{KIEM} see - \url{http://rtsys.informatik.uni-kiel.de/trac/kieler/wiki/Projects/KIEM}
\section{Outline of this Document}
\label{sec:intro/Outline}
The first part of this document is about the implemention of the Configuration plugin for the \ac{KIEM}.
The second part is about the implementation of an Automated Execution plugin for the \ac{KIEM}. Both
parts will have the same structure described below.
Each part starts with a detailed outline of the problems that are to be solved by this thesis.
The next section will be about the technologies used to solve the problem.
After that there will be a section about the concepts of how to solve the problem and some
design decisions that were made at a very early stage in the development.
Sections 5 and 6 are about the actual implementation with section 5 outlining the changes
that were made to the \ac{KIEM} plug-in itself and section 6 describing the newly created
\ac{KIEMConfig} plug-in.
The final section of each part will summarize the results of the thesis and outline
a few projects that could be based on it.







%%% Local Variables: 
%%% mode: latex
%%% TeX-master: "paper"
%%% End: 



\part{Configurations in the KIEM}
\include{ConfTask}
\include{ConfTechnologies}
\chapter{Concepts}
\label{chapter:ConfConcepts}
The solution to the problems outlined in chapter \ref{chapter:ConfTask} can be achieved with
the help of the Eclipse plug-in technology described in chapter \ref{chapter:ConfTechnology}.

\section{Configurations}
\label{section:ConfConceptsConf}
The first approach to saving additional configuration information in the execution files would
be to actually modify the format of those files and write the information into them.
This would most likely be the easiest approach but would destroy backward compatibility of
those files since the basic \ac{KIEM} would not know how to deal with the modified files.
The approach taken in this thesis is based on the list of DataComponents.

Each execution file carries a list of its own DataComponents and their properties to ensure
that the component are properly initialized the next time the execution is loaded. That list
can be loaded even if there are DataComponents contained in it that are not in the current
runtime configuration. The \ac{KIEM} will just show a warning but load the rest of the execution
anyway. 

These DataComponents basically just carry a list of KiemProperties which are
just (key, value) pairs. This makes them ideal for storing simple information like the value
of a text field.

To solve our problem we will just construct a new type of DataComponent and register it through
the extension point in the \ac{KIEM}. This ensures that the component can be added to any execution
file. The \ac{KIEM} ensures that all properties contained in our component will be saved with
the execution file and loaded the next time the file is opened. All we have to do is
find our component inside the DataComponent list, get the properties saved inside it
and set them inside the \ac{KIEM}.

% \begin{itemize}
%  \item .execution files carrying list of DataComponents, add configuration component
% automatically saved with file, doesn't break old files, new files are still
% can still be loaded without the plug-in
%  \item configurations saved anyway through the saveble view
% \end{itemize}

\subsection{Default Configuration}
\label{section:ConfConceptsDefaultConf}
\index{Default Configuration}
In order to provide a place to manage the default configurations we will be
using the Eclipse preference pages.

The root page for the \ac{KIEM} should contain the basic KIEM properties like
the aimed step duration, timeout and the view elements of the Configuration
plug-in.

\begin{figure}[MSPLayoutPreferencePage]
  \centering
  \includegraphics[scale=.5]{MSPLayoutPreferencePage.jpg}
  \caption[Layout Preference Page by Miro Sp\"onemann]%
  {Layout Preference Page by Miro Sp\"onemann\protect\footnotemark}
  \label{fig:MSPLayoutPreferencePage}
\end{figure}

Then we will construct another page for managing the different schedules
and their editors. For that we will adapt the LayoutPreferencePage by Miro Sp\"onemann
as seen in Figure \ref{fig:MSPLayoutPreferencePage} where we will just replace the 
layouts with the list of registered schedules.

In order to sort the schedules that match the currently opened editor they will
need to have user defined priorities which can be easily set up with the table
shown above.

The last page will be used to allow the user to set up his own properties and give them
default values.

The values entered in those pages will be stored inside the Eclipse Preference
Store.

%\begin{itemize}
% \item eclipse preference page architecture
% \item plug into KIELER preference page tree, create series of elements to
% configure item that are now only configurable through KiemView 
% \item use eclipse preference store to save values
% \item multiple pages for different aspects, user friendly, easier to maintain
%\end{itemize}

\section{Easier Configuration loading}
\label{section:ConfConceptsEasyLoading}
For easier configuration loading we will add two ComboBoxes to the existing
tool bar inside the \ac{KIEM} view.

One of them will display the list of recently used schedules the other one the
list of schedules that work for the currently active editor.

As soon as the user opens a new execution file through the normal workspace
view we will be notified of that event by the \ac{KIEM}. We will then create a new
schedule and store the path to the execution file in it along with the editor
that was opened at the time that the schedule was created. The new schedule will
also be added to the list of recently used schedules that we maintain through the 
use of the Eclipse preference store.

When the user selects one of the previously saved schedules in one of the
ComboBoxes we will retrieve the saved path and offer it to the KiemPlugin to
load it in the hopes that the execution file is still in the same location and
wasn't deleted, renamed or moved by the user.

% \begin{itemize}
%  \item keep a list of recently opened schedules in the preference store
%  \item allow access to that list through either menus or the KIEM itself
%  \item pass the saved path to KIEMPlugin to load it
%  \item store a list of editors that work for each .execution file together with their path
%  \item determine which editor is open and find matching files
% \end{itemize}
\include{ConfKiemChanges}
\include{kiemConfig}
\chapter{Conclusion}
\label{chapter:Conclusion}
\section{Results}
\begin{itemize}
 \item problems solved
\end{itemize}

\section{Further improvements}
\begin{itemize}
 \item allow Objects (Serializable) to be saved
 \item improve dynamic layout
 \item refactore entire execution manager to use the runtime mechanism
\end{itemize}

\part{Automated Execution in the KIEM}
\include{AutoTask}
\chapter{Used technologies}
\label{chapter:AutoTechnologies}

The technologies used in this thesis are basically the same as in
the first part (see Chapter \ref{chapter:ConfTechnology}).

\section{Eclipse WorkbenchJob API}
\label{section:AutoTechJob}
Since we don't want the UI to block while we are doing our automated
execution run we will use the Eclipse Job API to run our
automated execution thread.
The Job API is specifically designed to accommodate very long running
tasks which makes it perfect for our purposes since an execution
run with several model files and hundreds of trace files might
a whole night.
An example for the use of jobs in the normal Eclipse architecture
is the SVN commit operation seen in Figure \ref{fig:SVNCommit}.

\begin{figure}[SVNCommit]
  \centering
  \includegraphics[scale=.4]{SVNCommit.png}
  \caption[The SVN commit job]%
  {The SVN commit job\protect\footnotemark}
  \label{fig:SVNCommit}
\end{figure}



\section{Eclipse Wizards}
\label{section:AutoTechWizards}
To allow the user to set up the execution run we will be using the
Eclipse Wizard API.
Wizards are used to guide the user through the process of creating complex items by taking
the information in a structured way and then generating the item from it.
One example inside the Eclipse Architecture is the Java Class Creation Wizard (see Figure \ref{fig:ClassWizard})
In theory it is possible to just open a text file and enter all the information manually.
However if the wizard is used the user only has to select the class he wants to extend
and the interfaces he wants to implement, activate one check box and then the wizard will
create the class body, all required methods and comments for each element (see Listing \ref{fig:classCreationGenerated})
This makes it very easy for even inexperienced users to create new classes without knowing
the exact syntax.

\begin{figure}[ClassWizard]
  \centering
  \includegraphics[scale=.4]{ClassWizard.png}
  \caption[The Class Creation Wizard]%
  {The Class Creation Wizard\protect\footnotemark}
  \label{fig:ClassWizard}
\end{figure}

\lstset{
  language=Java,
}
\begin{lstlisting}[caption={Code generated by the wizard},label={fig:classCreationGenerated}]
package test;

import org.eclipse.jface.action.ControlContribution;
import org.eclipse.swt.widgets.Composite;
import org.eclipse.swt.widgets.Control;

/**
 * @author soh
 */
public class MyClass extends ControlContribution implements Runnable {

    /**
     * Creates a new MyClass.java.
     * 
     * @param id
     */
    public MyClass(String id) {
        super(id);
    }

    /**
     * {@inheritDoc}
     */
    @Override
    protected Control createControl(Composite parent) {
        return null;
    }

    /**
     * {@inheritDoc}
     */
    public void run() {
    }

    /**
     * @param args
     */
    public static void main(String[] args) {

    }
}
\end{lstlisting}



\chapter{Concepts}
\label{chapter:AutoConcepts}

\section{Setting up an Automated Run}
\label{section:AutoConceptsSetup}
There are several possibilities of how to solve the problem of accumulating
large amounts of information prior to a long running action.
The first possibility would be to have the user enter the paths to the 
necessary files in text files, parse those files and start a run with
the parsed information. While this is a good method for performing
static runs from a console environment it has several disadvantages
inside the GUI of an Eclipse RCA:
\begin{itemize}
 \item Manually entered file names in a text file are prone to have erroneous information.
It is very hard to manually enter the correct file name of any file and the entered location
only works on one OS. Aside from that it takes a long time to manually enter the possible wast
amount of files used.
 \item There is no way to quickly adjust the file if other models or execution files should be used.
 \item It also means more files cluttering up the workspace.
 \item It is not very intuitive and the user has to know the exact syntax that the execution needs.
\end{itemize}

A far easier approach is the selection of the files through the use of a dialog.
Here the first option is to write a new Dialog from scratch. While this option
ensures flexibility since only the elements that are really needed are on it in
exactly the way they are needed it has also a few downsides:
\begin{itemize}
 \item It involves a lot of work since every widget has to be manually placed on the dialog.
 \item It involves even more work to get the layout of the dialog just right.
\end{itemize}

The easiest way it to use one of the dialogs provided by Eclipse specifically the wizard type dialog.
Eclipse itself uses a host of wizards as explained in [ref here].
The wizard has several advantages over the other methods explained here:
\begin{itemize}
 \item Even inexperienced users can be guided to set up a valid execution run.
 \item The entered information is most likely valid since the wizard only displays valid files.
 \item It is quicker to program and easier to adjust than any of the other methods.
\end{itemize}


\section{Input for the Automation}
\label{section:AutoConceptsInput}
In order to input information into the DataComponents the first decision has to be in
what form the information will be supplied.
The chosen form is that of a list of key, value pairs. It allows for the most
flexiblility while still being very generic and simple to read and write on.
This list of properties will at least include the path to the model file in order
for components to be executed with several different model files without having
to alter the code between runs.

The next question is how the components will get the information.

The first possibility would be to have the component ask the plug-in for the information
in question. The upside of this would be that components are sure to get all the information
they need before the execution can start since they can keep asking for it. However this would
likely mean that the component would have to ask multiple times as they cannot know when
the required information will be available which constitutes additional workload. Furthermore
this situation would likely mean that multiple components might request information
at the same time. This means that there would be the need for substantial synchronization mechanisms
to ensure consistency of data.

Therefore the way chosen in this thesis is that the Execution Manager will inform interested components
about all properties that were accumulated and then starts the execution run. This ensures
that a run is started in any event and keeps communication between the components and the manager simple.

\section{Automate the execution}
\label{section:AutoConceptsExecution}
Automating the execution itself requires the plug-in to interact with the KIEM.
There is already an API defined for loading an execution file by supplying a path so that
is what will be used in this project.
Then it is neccasary to initialize the execution and step through it using the API
methods provided in the Execution. For this the EventListener extension point of the 
KIEM can also be used in order to determine when a step has finished executing and
a new one can be dispatched.
After the execution is finished all components should be called again to be given
a chance to provide some information for the display in the next step. This information
will be gathered in the same form and way as described in [ref to pref section].


\section{Output of the results of the execution}
\label{section:AutoConceptsOutput}
On the subject of displaying the information again several options
are available.


The last objective is to display the information in a meaningful way.
This should involve at least two methods of output:
\begin{enumerate}
  \item A formatted string possibly in an XML fashion that can be parsed and
  used by other plug-ins for automatic analysis.
  \item Some graphic component that will display the information in way that is
  easy to read for most users.
\end{enumerate}
\chapter{Code changes in KIEMPlugin}
\label{chapter:AutoKiemChanges}
\begin{itemize}
 \item interfaces and API changes to allow access to execution
 \item changes to load values rather than hard coded defaults
\end{itemize}

\section{Schema files and Interfaces}
\begin{itemize}
 \item event listener mentioned in part I, for listening to KIEM execution
 events
 \item interfaces added to KIEM itself to avoid components breaking when
 automated plug-in is not loaded
\end{itemize}

\section{Automated Component}
An automated component is any DataComponent that wants to interact with
the automated execution plug-in. As seen in the diagram automated components
have to implement three methods:

\subsection{provideProperties()}
This method enables components to receive information prior to each execution
run. The list is implemented as an array of key, value pairs stored inside
KiemProperty objects.
At the every least the list contains the location of the model file and the
index of the currently running iteration (that is how many time the current
model has already been executed with the current execution file).
This allows components to load additional files that are always in the
same path as the execution file and determine which of those to load
based on the iteration index.

\subsection{wantAnotherStep()}
This method is called after every step of the execution manager.
All components are asked if they want to perform another step and if
one components answers with TRUE the execution manager will perform another
step.
This makes it unnecessary to know the exact number of steps needed for
the execution before the execution starts.

\subsection{wantsAnotherRun()}
This method is the equivalent for the wantAnotherStep() method in the context
of entire execution runs.
After no component wants to perform another step all components are asked
if they want to perform another run. If one component answers with TRUE
the entire simulation is stopped and then started again with the same
model file and execution file but an incremented iteration index.
This can for example be used to execute the same model with a different
set of trace files by just naming the trace files the same way as the
model files with a number at the end.

\section{Automated Producer}
This interface extends the AutomatedComponent interface.
In addition to the inherited methods it provided one additional method.
This method is called after an iteration has finished and asks the components
if they want to publish any information about the results of their execution.
This information is gathered by the plug-in and the accumulated results
are either passed to the calling plug-in or displayed in the
specially designed view (see chapter about the view).

\chapter{The Automated Executions Plug-in}
- new plugin
- handles the setup, control flow and display of automated execution
- consists of 3 parts explained in detail below
- wizard, manager, view
\section{The wizard}
\subsection{File selection page}
- wizard is used to set up the execution run
- extends ResourceImportWizard for displaying a folder/file structure for selecting files from
- easily usuable, select whole folders, filter file types
- can be given an initial selection, on close will save the selection, store it in preference store
and restore it on load
- additional dialog for selecting execution files that are not in the workspace but imported
- for simplicity assume that files ending .execution are execution files and all other selected
ones are model files, wizard can not check if valid since formats are not known
- only allow user to proceed if at least one execution and one model file is selected
\subsection{Property Setting page}
- set up the additional arguments passed to the execution
- simple adding and removing of key, value pairs
- same as file page, on close results are saved to preference store and restored for initial
selection on next open
\subsection{Processing the information}
- gather execution files and model files from file selection page
- gather properties from property page
- store information for next open
- invoke the execution manager

\section{The Execution manager and job}
- handles control flow during the automation
- takes information from either call through the API or wizard

\subsection{Execution Manager}
- handles the overall control flow
- takes the execution files, model files and properties as argument
- if progress monitor is registered it is informed about the progress of the evaluation
- The control flow:
\begin{itemize}
 \item iterate over all execution files
 \item open execution file
 \item tell view to set up display for the first execution file
 \item iterate over all model files
 \item get first model file from list
 \item ask components how many more runs they need, take maximum and perform runs before asking again
 \item pass model file, execution file and index of iteration
 \item initialize the execution
 \item pass properties to components, at least receive model file and iteration
 \item start worker thread that listens for monitor canceling, step timing out
 \item ask components how many more steps they need, take maximum and perform steps before asking again
 \item perform one step, lock self inside semaphore, stay locked until either worker thread or event listener notifies (step done)
 \item when no component wants more steps pause
 \item gather information from all IAutomatedProducers
 \item tell view to show information for this iteration
 \item stop execution inside the KIEM and perform cleanup
 \item proceed to next iteration
 \item inform monitor of progress
 \item proceed to next model
 \item proceed to next execution
 \item when done inform monitor of done and terminate the job
\end{itemize}



\subsection{Execution Job}
- workbenchjob with progressmonitor
- used to display the progress in the progress view and a dialog with progress bar
- long running task, doesn't want to block the rest of the workbench 
- triggers execution in the manager

\section{The View}
- displays the information in a structured way
- start a new table for each execution file
- one row for each iteration with each model file
 - prerequisite needed here: always the same outputs throughout the entire execution file
- first columns display model file name, iteration index and current status

\subsection{Toolbar}
- button to start the wizard
- button for clearing the view
- text field showing the step that was just processed

\chapter{Conclusion}
\label{chapter:AutoConclusion}
\section{Results}

\section{Further improvements}
\begin{itemize}
 \item implement macro step support as soon as KIEM does
\end{itemize}




\appendix

\backmatter
%\include{glossary}
\include{index}
\begin{thebibliography}{cmot-dt}
 \bibitem{cmot-dt}Christian Motika, Semantics and Execution of
 Domain Specific Models, 2009.
 \bibitem{eclipseOverview}Object Technology International, Inc. Eclipse Platform Technical Overview,
2003.
 \bibitem{eclipsePlugins}Eric Clayberg and Dan Rubel. Eclipse Plug-ins. Addison Wesley, 2009.
\end{thebibliography}

%\chapter{Bedienungsanleitung}\index{Bedienungsanleitung}
Um mit der Uhr arbeiten zu k"onnen muss man als erstes eine Batterie
einlegen. Die Uhr wechselt dann sofort in den Macrozustand \zst{alive}, und
dort in den Unterzustand \zst{displays}. Von nun an werden Sekunden Minuten
und Stunden auf dem LCD-Display angezeigt. 


\begin{description}
\item[Dr"ucken von \kn{Knopf d}:]
M"ochte der Benutzer der Uhr sich nun  das datum ausgeben lassen, dann dr"uckt
er den \kn{Knopf d}. Die Uhr wechselt nun in den Zustand \zst{date} und das 
Datum
erscheint auf der LCD-Anzeige. 

\item[Dr"ucken von \kn{Knopf a}:]
Um in das uhr-Men"u zu gelangen muss man nur den \kn{Knopf a} dr"ucken. 
Folgende mehrfachbenutzung ist m"oglich:
\begin{description}
\item[Bei einfachem Dr"ucken von\kn{ Knopf a}]
gelangt man in das Men"u von Alarm1. Mit Hilfe des \kn{Knopfes d} wird der 
Alarm1
in den Zustand \zst{on} oder durch erneutes Dr"ucken in den Zustand  \zst{off}
geschaltet. 
Befindet sich der Alarm1 im Zustand \zst{on}, dann erscheint eine kleine
Glocke mit einer 1  oben links in der Ecke der Uhr. 
Befindet sich nun der Alarm1 im Zustand \zst{off}, dann erlischt die kleine
Glocke wieder. 

Um den Alarm zu stellen muss der Benutzer den \kn{Knopf c} dr"ucken. Nun kann 
man mit dem \kn{Knopf d} die Stunden erh"ohen. Ist die gew"unschte einstellung
vorgenommen, dann gelangt man "uber das dr"ucken von \kn{Knopf c} zum Stellen 
der Zehnerstelle der Minuten. Nach fertigem einstellen gelangt man wieder "uber
einen Druck auf \kn{Knopf c} zum Stellen der Minuten. Sind nun alle 
Einstellungen vorgenommen worden, dann gelangt man mit Hilfe des 
\kn{Knopfes c} zur"uch ins Men"u von Alarm1. 
Um wieder in den \zst{time}-Zustand zu gelangen, in dem die Uhrzeit
angezeigt wird muss man noch viermal den Knopf a dr"ucken. 

\item[Bei zweifachem Dr"ucken von \kn{Knopf a}]
gelangt man in das Men"u von Alarm2. Das Men"u von Alarm2 ist genauso
aufgebaut, wie das Men"u von Alarm1 (s.o.). Mit der ausnahme, dass man beim
verlassen des Zustandes durch dreifaches Dr"ucken in den Zustand \zst{time}
gelangt. 
 
\item[Bei dreifachem Dr"ucken von \kn{Knopf a}]
gelangt man in das Men"u des Zustands \zst{Chime}. Mit Hilfe des \kn{Knopfes d}
kann man nun einstellen, dass die Uhr alle volle Stunde einmal  alarm schlagen
soll. Durch erneutes Dr"ucken des Knopfes d wird diese Funktion wieder
ausgeschaltet. Ist der Zustand \zst{Chime} aktiv, dann erscheint auf dem
Uhr-\eng{display} eine kleine gelbe Glocke. Ist der Zustand inaktiv, dann 
erlischt sie wieder.  
Um wieder in den \zst{time}-Zustand zu gelangen muss man noch zweimal den 
Knopf a dr"ucken. 

\item[Bei vierfachem Dr"ucken von Knopf a] gelangt man in das Stopuhr-Men"u.
  Die LCD-Anzeige erweitert sich automatisch um die M"oglichkeit nun auch \cs
  anzeigen zu k"onnen. Durch das Dr"ucken von \kn{Knopf c} startet man die 
Stopuhr. Um nun die Zeit zu stoppen gen"ugt ein erneutes Dr"ucken des 
\kn{Knopfes c}.  M"ochte man die Stopuhr wieder auf Null setzen, dann muss 
man den \kn{Knopf d}  dr"ucken.  In den Zustand \zst{time} gelangt man durch 
einfaches Dr"ucken  von \kn{Knopf a}.
\end{description}

\item[Dr"ucken von \kn{Knopf b}:]
Die LCD-Anzeigen Beleuchtung der Uhr erscheint durch das Dr"ucken von \kn{Knopf
b}. Wird nun nocheinmal dieser Knopf gedr"uckt, dann erlischt das Licht
wieder. 
\item[Batterie:]
Ist die Batterie der Uhr schwach (Das Signal \signal{Batt\_weakens} wird
emittiert), dann beginnt die Uhr zu blinken. Wird die Batterie
herausgenommen oder erlischt, dann bleibt die Uhr stehen. 
\end{description}

%%% Local Variables: 
%%% mode: latex
%%% TeX-master: "paper"
%%% End: 

%
\chapter{C++-Code f�r die GUI}
\label{sec:code}

\section{Beschreibung}



\begin{description}

\item[\texttt{haupt.cpp}:]
In der Datei haupt.cpp ist die Klasse \code{Nebenbei}, sowie die Instantiierung
des \eng{Widgets} beschrieben. Die Klasse \code{Nebenbei} realisiert die
Nebenl�ufigkeit des Programms. Sie besteht aus einem Konstruktor, der
bekanntlich genauso hei�t wie seine Klasse und der Methode \code{run}. In 
der Methode \code{run} wird der parallel zum \gui\ laufende Code
definiert. Hier wird als erstes die Digitaluhr instantiiert. In der
\code{while}-Schleife werden die \eng{Automation Engine}- und
\eng{Input}-Funktionen aufgerufen. Mit der Abfrage, ob der Wert der zum
\eng{Input}-Signal passenden Variable gleich zwei ist wird �berpr�ft, ob der
Benutzer dieses Signal emittieren m�chte. 

\item[\texttt{Gui.h}:]
In dieser \eng{Header}-Datei sind die Klasse \code{GraficWidget}, ihre
\eng{Slots} und ihre Methoden definiert. Au�erdem sind hier die globalen
Variablen definiert, die die Kommunikation zwischen \eng{Thread} und
\gui\ realisieren. 

\end{description}


\section{Der Code}

\begin{landscape}
  \lstfile{code/}{haupt.cpp}
  \lstfile{code/}{Gui.cpp}
\end{landscape}


%% Alternativ:

%% \subsection{\texttt{haupt.cpp}}

%% \VerbatimInput[%
%%   frame=topline,
%%   framesep=2mm,
%%   label=haupt.cpp,
%%   fontsize=\footnotesize,
%%   numbers=left]%
%%   {code/haupt.cpp}

%% \newpage

%% \subsection{\texttt{Gui.h}}

%% \VerbatimInput[%
%%   frame=topline,
%%   framesep=2mm,
%%   label=GUI.h,
%%   fontsize=\footnotesize,
%%   numbers=left]%
%%   {code/Gui.h}





%%% Local Variables: 
%%% mode: latex
%%% TeX-master: "paper"
%%% End: 



\end{document}

%%% Local Variables: 
%%% mode: pdflatex
%%% TeX-master: paper.tex
%%% End: 